\begin{frame}
        \frametitle{Overview}
        \begin{itemize}
                  \setlength{\itemindent}{1cm}
                \item[\textbf{Title:}] Demand-Driven Cycamore Archetypes 
                \item[\textbf{PI:}] Anthony Scopatz, University of South 
                        Carolina\footnote{Anthony departed academia in year 2 
                        of the project. The PIship was transferred to Travis 
                        Knight at USC}
                \item[\textbf{Co-PI:}] Kathryn Huff, University of Illinois at 
                        Urbana-Champaign
                \item[\textbf{Start:}] October, 2016
                \item[\textbf{End:}] October, 2017
                \item[\textbf{Objectives:}] Develop an in situ demand 
                        driven development schedule calculation through 
                        non-optimizing, deterministic-optimizing, and 
                        stochastic-optimizing algorithms as Cyclus archetypes. 
                        Demonstrate these new archetypes in program-supporting 
                        fuel cycle scenarios.
        \end{itemize}
\end{frame}


\begin{frame}
        \frametitle{Quick Statistics}
        \begin{block}{Publications Affiliated with this Work}
                \begin{itemize}
                  \setlength{\itemindent}{3cm}
                        \item[\textbf{Journal Articles}] 3 (1 upcoming)
                        \item[\textbf{Full Conference Papers}] 3 (2 upcoming) 
                        \item[\textbf{Conference Summaries}] 7
                        \item[\textbf{Technical Reports}] 2 (1 upcoming)
                        \item[\textbf{Theses}] 1MS (2 upcoming)
                \end{itemize}
        \end{block}

                \begin{block}{Students Supported}
                        The funding supported graduate students and 
                        occasional undergraduates at UIUC.
                        \textbf{Jin Whan Bae} recieved his MS and is now at 
                        ORNL purusing Cyclus usability.
                        \textbf{Gwendolyn Chee} is writing an MS thesis related 
                        to this work and related work conducted at ANL with Bo Feng.
                        Undergraduate \textbf{Louis Kissinger} is a 
                        baccalaureate researcher this year in MCS at ANL.
                        Others include \textbf{Roberto Fairhurst}, \textbf{Gyu 
                        Tae Park}, \textbf{Snehal Chandan}, and \textbf{Aditya 
                        Bhosale}.
                \end{block}
\end{frame}

\begin{frame}
        \frametitle{Detailed Schedule}
\begin{block}{2016}
        \begin{itemize}
       \item[$\checkmark$] Literature review of appropriate predictive algorithms.
       \item[$\checkmark$] Add stop and restart capabilities to Cyclus (bonus: HPC deployment addition).
        \end{itemize}
\end{block}
\begin{block}{2017}
        \begin{itemize}
       \item[$\checkmark$] Identify and rectify non-algorithmic capability gaps (e.g. specific fuel cycle process archetypes) necessary for transition simulation.
       \item[$\checkmark$] Create d3ploy
       \item[$\checkmark$] Add toolkit additions related for geospatial information
       \item[$\checkmark$] Implement non-optimizing (NO) methods in d3ploy.
        \end{itemize}
\end{block}
\end{frame}


\begin{frame}
        \frametitle{Detailed Schedule}
\begin{block}{2018}
        \begin{itemize}
       \item[$\checkmark$] Design numerical experiments (tests) for verifying Deterministic-Optimizing (DO) algorithms in the context of key transitions.
       \item[$\checkmark$] Implement Deterministic Optimizing (DO) methods in d3ploy.
       \item[$\checkmark$] Design numerical experiments (test) for verifying Stochastic-Optimizing (SO) algorithms in the context of key transitions. 
        \end{itemize}
\end{block}
\begin{block}{2019}
        \begin{itemize}
       \item[$\checkmark$] Implement Stochastic Optimizing (SO) methods in d3ploy.
       \item[$\checkmark$] Add additional capabilities to the predictive methods. (Buffers, reprocessing complexity handing)
       \item[$\checkmark$] Demonstrate and compare the new capability in the context of the evaluation groups the EG 23, 24, 29, 30
        \end{itemize}
\end{block}
\end{frame}


\begin{frame}
  \frametitle{Goal of the Project}
  % a comment
        \begin{block}{Main Objective}
              To improve usability of Cyclus for transition scenarios.
        \end{block}
        

        \begin{block}{Main Challenge}
              Deploying reactors to meet power demand is trivial, and existed 
                in the earliest versions of Cyclus.
              \textbf{Automated, predictive deployment and decommissioning of 
                other facilities is more complex.} These include mining, 
                milling, enrichment, fuel 
                fabrication, reprocessing, and others. 

              For example, a balanced closed fuel cycle may require ensuring 
                that there is enough fast reactor fuel for their operation and 
                may drive deployment of a fleet of light water reactors.
        \end{block}
\end{frame}

\begin{frame}
        \frametitle{Method}
        \begin{block}{Because Cyclus is Agent-Based}
                \begin{itemize}
                        \item Its regions and institutions have the agency to dynamically make and alter deployment decisions.
                        \item Each agent can make their own predictions of the future based on current and past performance of the simulation.
                \end{itemize} 
                
                We embedded advanced time series prediction algorithms to automatically 
                deploy fuel cycle facilities for the user. This was implemented 
                in \textbf{\texttt{d3ploy}}, an Institution agent. 
        \end{block}
\end{frame}

%\begin{frame}
%  \frametitle{Goal of the Project}
%  % a comment
%  \begin{itemize}
%    \item[$\bullet$] Three types of methods were looked at for doing the prediction.
%    \item[$\bullet$] Non-optimizing methods. These methods included moving average
%                     , autoregressive moving average (ARMA), and autoregressive
%                     heteroskidasticity (ARCH) models. 
%    \item[$\bullet$] Deterministic models. These models use a methodology that will
%                     always return the same answer given a unique input. This includes
%                     Fast Fourier Transforms, Exponential Smoothing, Holt-Winters,
%                     Polynomial regression. 
%    \item[$\bullet$] Stochastic models. With stochastic models, the output is based
%                     on randomly sampling models to determine the behavior of the 
%                     next time step. The method implimented here is a machine learning
%                     seasonal method. 
%   \end{itemize}        
%\end{frame}
%

  \begin{frame}
    \frametitle{Motivation}

    \textbf{Gap in capability: User must define when support facilities are deployed}

    \begin{figure}[htbp!]
      \begin{center}
        \includegraphics[width=0.8\textwidth]{images/user-deploy}
      \end{center}
            \caption{User defined Deployment Scheme }
    \end{figure}

    \textbf{Bridging the gap: Developed demand-driven deployment capability in \Cyclus. This capability is named \deploy.}

    \begin{figure}[htbp!]
      \begin{center}
        \includegraphics[width=0.8\textwidth]{images/auto-deploy}
      \end{center}
            \caption{Demand Driven Deployment Scheme}
    \end{figure}

  \end{frame}
